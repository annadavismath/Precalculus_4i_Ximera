\documentclass{ximera}

\begin{document}
	\author{Stitz-Zeager}
	\xmtitle{The Circular Functions: Tangent, Secant, Cosecant, and Cotangent}


\mfpicnumber{1}

\opengraphsfile{TheOtherCircularFunctions}

\setcounter{footnote}{0}

\label{TheOtherCircularFunctions}

In section \ref{TheCircularFunctionsSineandCosine},  we extended the notion of $\sin(\theta)$ and $\cos(\theta)$  from acute angles to any angles using the coordinate values of points on the Unit Circle. In total, there are six circular functions, as listed below.
\smallskip


%% \colorbox{ResultColor}{\bbm

\begin{definition} \label{circularfunctions}  \textbf{The Circular Functions:} Suppose an angle $\theta$ is graphed in standard position. 

\smallskip

Let $P(x,y)$ be the point of intersection of the terminal side of $P$ and the Unit Circle. 


 \begin{itemize}

\item The \index{sine ! of an angle} \textbf{sine} of $\theta$, denoted $\sin(\theta)$, is defined by $\sin(\theta) = y$. \vphantom{$\dfrac{y}{x}$}

\item The \index{cosine ! of an angle} \textbf{cosine} of $\theta$, denoted $\cos(\theta)$, is defined by $\cos(\theta) = x$. \vphantom{$\dfrac{y}{x}$}

\item The \index{tangent ! of an angle} \textbf{tangent} of $\theta$, denoted $\tan(\theta)$, is defined by $\tan(\theta) = \dfrac{y}{x}$, provided $x \neq 0$. \vphantom{$\dfrac{y}{x}$}

\item The \index{secant ! of an angle} \textbf{secant} of $\theta$, denoted $\sec(\theta)$, is defined by $\sec(\theta) = \dfrac{1}{x}$, provided $x \neq 0$. \vphantom{$\dfrac{y}{x}$}

\item The \index{cosecant ! of an angle} \textbf{cosecant} of $\theta$, denoted $\csc(\theta)$, is defined by $\csc(\theta) = \dfrac{1}{y}$, provided $y \neq 0$. \vphantom{$\dfrac{y}{x}$}

\item The \index{cotangent ! of an angle} \textbf{cotangent} of $\theta$, denoted $\cot(\theta)$, is defined by $\cot(\theta) = \dfrac{x}{y}$, provided $y \neq 0$. \vphantom{$\dfrac{y}{x}$}

\end{itemize}

\end{definition}

\smallskip

%% \ebm}

\smallskip

While we left the history of the name `sine' as an interesting research project in Section \ref{TheCircularFunctionsSineandCosine},we take a slight detour here to explain the origin of the names `tangent' and `secant.'  

\smallskip

Consider the acute angle $\theta$  in standard position sketched in the diagram below. 

\smallskip

\begin{center}

\begin{mfpic}[25]{-1}{7}{-1}{7}
\axes
\drawcolor[gray]{0.7}
\parafcn{0,90,5}{3*dir(t)}
\drawcolor{black}
\arrow \parafcn{5, 55, 5}{0.75*dir(t)}
\tlabel[cc](0.75,0.5){\scriptsize $\theta$}
\point[4pt]{(0,0), (1.5, 2.5981), (3,5.196), (3,0), (1.5,0)}
\tlabel(6.75,-0.5){\scriptsize $x$}
\tlabel(0.25,6.75){\scriptsize $y$}
\tlabel(0.25,3.1){\scriptsize $1$}
\tlabel(-0.5,-0.5){\scriptsize $O$}
\tlabel(2.75,-0.5){\scriptsize $B(1,0)$}
\tlabel(1.25, -0.5){\scriptsize $A(x, 0)$}
\xmarks{0 step 3 until 3}
\ymarks{0 step 3 until 3}
\polyline{(3,0), (3,5.196)}
\polyline{(2.75,0), (2.75, 0.25), (3,0.25)}
\polyline{(1.5,0), (1.5, 2.5981)}
\polyline{(1.25,0), (1.25,0.25), (1.5,0.25)}
\tlabel(1.75,2.6){\scriptsize $P(x,y)$}
\tlabel(3.25,5.25){\scriptsize $Q(1,y') = (1, \tan(\theta))$}
\penwd{1.25pt}
\arrow \reverse \arrow  \polyline{(7,0), (0,0), (4,6.9282)}
\end{mfpic} 

\end{center}

\smallskip

As usual, $P(x,y)$ denotes the point on the terminal side of $\theta$ which lies on the Unit Circle, but we also consider  the point $Q(1,y')$,  the point on the terminal side of $\theta$ which lies on the vertical line $x=1$. 

\smallskip

The word `tangent' comes from the Latin meaning `to touch,' and for this reason, the line $x=1$ is called a \textit{tangent} line to the Unit Circle since it intersects, or `touches', the circle at only one point, namely $(1,0)$.  

\smallskip

Dropping perpendiculars from $P$ and $Q$ creates a pair of similar triangles $\Delta OPA$ and $\Delta OQB$.  Hence the corresponding sides are proportional.  We get  $\frac{y'}{y} = \frac{1}{x}$ which gives  $y' = \frac{y}{x} = \tan(\theta)$.

\smallskip


 We have just shown that for acute angles $\theta$, $\tan(\theta)$ is the $y$-coordinate of the point on the terminal side of $\theta$ which lies on the line $x = 1$ which is \textit{tangent} to the Unit Circle. 
 
 \smallskip
 
 The word `secant' means `to cut', so a secant line is any line that `cuts through' a circle at two points.\footnote{Compare this with the definition given in Section \ref{AverageRateofChange}.}  The line containing the terminal side of $\theta$ (not just the terminal side itself) is one such secant line since it intersects the Unit Circle in Quadrants I and III.   
 
 \smallskip
 
 With the point $P$ lying on the Unit Circle, the length of the hypotenuse of $\Delta OPA$ is $1$. If we let $h$ denote the length of the hypotenuse of $\Delta OQB$, we have from similar triangles that $\frac{h}{1} = \frac{1}{x}$, or $h = \frac{1}{x} = \sec(\theta)$.  
 
 \smallskip
 
 Hence for an acute angle $\theta$, $\sec(\theta)$ is the length of the line segment which lies on the secant line determined by the terminal side of $\theta$ and `cuts off' the tangent line $x=1$.  
 
 \smallskip
 
As we mentioned in Definition \ref{sinecosineunitcircledefn}, the `co' in `cosecant' and `cotangent' tie back to the concept of `co'mplementary angles and is explained in detail in Section \ref{MoreTrigonometricIdentities}.

 \smallskip

 Not only do these observations help explain the names of these functions, they serve as the basis for a fundamental inequality needed for Calculus which we'll explore in the Exercises.

\smallskip 

Of the six circular functions, only sine and cosine are defined for all angles $\theta$.  Since $x = \cos(\theta) $ and $y = \sin(\theta) $ in Definition \ref{circularfunctions}, it is customary to rephrase the remaining four circular functions  Definition \ref{circularfunctions} in terms of sine and cosine.  


\smallskip

%% \colorbox{ResultColor}{\bbm

\begin{theorem} \label{recipquotid}  \textbf{Reciprocal and Quotient Identities:} \index{Reciprocal Identities} \index{Quotient Identities} 

\begin{itemize}

\item $\sec(\theta) = \dfrac{1}{\cos(\theta)}$, provided $\cos(\theta) \neq 0$;  if $\cos(\theta) = 0$, $\sec(\theta)$ is undefined.

\item $\csc(\theta) = \dfrac{1}{\sin(\theta)}$, provided $\sin(\theta) \neq 0$;  if $\sin(\theta) = 0$, $\csc(\theta)$ is undefined.

\item $\tan(\theta) = \dfrac{\sin(\theta)}{\cos(\theta)}$, provided $\cos(\theta) \neq 0$;  if $\cos(\theta) = 0$, $\tan(\theta)$ is undefined.

\item $\cot(\theta) = \dfrac{\cos(\theta)}{\sin(\theta)}$, provided $\sin(\theta) \neq 0$;  if $\sin(\theta) = 0$, $\cot(\theta)$ is undefined.

\end{itemize}

\end{theorem}

%% \ebm}

\smallskip

We call the equations listed in Theorem \ref{recipquotid}  \index{identity ! equation}\index{equation ! identity}\textbf{identities} since they are relationships which are true regardless of the values of $\theta$.  This is in contrast to \index{conditional equation}\index{equation ! conditional}\textbf{conditional equations} such as $\sin(\theta) = 1$ which are true for only \textbf{some} values of $\theta$.  We will study identities more extensively in Sections \ref{FundamentalTrigonometricIdentities} and \ref{MoreTrigonometricIdentities}.

\smallskip

While the Reciprocal and Quotient Identities presented in Theorem \ref{recipquotid} allow us to always reduce problems involving secant, cosecant, tangent and cotangent to problems involving sine and cosine, it is not always convenient to do so.\footnote{As we shall see shortly, when solving equations involving secant and cosecant, we usually convert back to cosines and sines.  However, when solving for tangent or cotangent, we usually stick with what we're dealt.}  It is worth taking the time to memorize the tangent and cotangent values of the common angles summarized below.

\smallskip

\begin{center}

\textbf{Tangent and Cotangent Values of Common Angles}

\vspace{-.25in}

\setlength{\extrarowheight}{4pt}

\[ \begin{array}{|c|c||c|c|} \hline
 \theta (\mbox{degrees}) &  \theta (\mbox{radians}) & \tan(\theta) & \cot(\theta) \\ \hline
0^{\circ} & 0 & 0 & \text{undefined} \\ \hline
30^{\circ} & \frac{\pi}{6} & \frac{\sqrt{3}}{3} & \sqrt{3} \\ [2pt] \hline
45^{\circ} & \frac{\pi}{4} & 1 & 1 \\ [2pt] \hline
60^{\circ} & \frac{\pi}{3} & \sqrt{3} & \frac{\sqrt{3}}{3} \\ [2pt] \hline
90^{\circ} & \frac{\pi}{2} & \text{undefined} & 0 \\ [2pt] \hline
\end{array} \]

\setlength{\extrarowheight}{2pt}

\end{center}

Coupling Theorem \ref{recipquotid} with the Reference Angle Theorem, Theorem \ref{refanglethm}, we get the following.

\smallskip

%% \colorbox{ResultColor}{\bbm

\begin{theorem} \label{genrefanglethm} \textbf{Generalized Reference Angle Theorem.}  The values of the circular functions of an angle, if they exist, are the same, up to a sign, of the corresponding circular functions of its reference angle. 

\smallskip

More specifically, if $\alpha$ is the reference angle for $\theta$,  then: 

\smallskip

  \centerline{$\sin(\theta) = \pm \sin(\alpha)$,  $\cos(\theta) = \pm \cos(\alpha)$,  $\tan(\theta) = \pm \tan(\alpha)$}
  
  \centerline{and}
  
    \centerline{$\sec(\theta) = \pm \sec(\alpha)$,  $\csc(\theta) = \pm \csc(\alpha)$,  $\cot(\theta) = \pm \cot(\alpha)$}
    
    \smallskip


where the choice of the ($\pm$) depends on the quadrant in which the terminal side of $\theta$ lies.\index{Reference Angle Theorem ! for the circular functions}

\end{theorem}

%% \ebm}

\smallskip

It is high time for an example.


\begin{example} \label{circularfunctionsex}  $~$

\begin{enumerate}

\item  Find the exact value of the following, if it exists:

\begin{multicols}{4}

\begin{enumerate}

\item  $\sec\left(60^{\circ}\right)$

\item  $\csc\left(\frac{7 \pi}{4} \right)$

\item  $\tan(225^{\circ})$

\item  $\cot\left(-\frac{7 \pi}{6} \right)$

\end{enumerate}

\end{multicols}

\item Find all angles which satisfy the given equation.   

\begin{multicols}{4}

\begin{enumerate}

\item  $\sec(\theta) =2$

\item  $\csc(\theta) = -\sqrt{2}$

\item  $\tan(\theta) = \sqrt{3}$

\item  \label{cotangentisnegativeone} $\cot(\theta) = -1$.

\end{enumerate}

\end{multicols}

\end{enumerate}

{\bf Solution.}  

\begin{enumerate}

\item \begin{enumerate}

\item  According to Theorem \ref{recipquotid},   $\sec\left(60^{\circ}\right) = \frac{1}{\cos\left(60^{\circ}\right)}$. Hence,  $\sec\left(60^{\circ}\right) = \frac{1}{(1/2)} = 2$.

\item  Since $\sin\left( \frac{7\pi}{4}\right) = - \frac{\sqrt{2}}{2}$,  $\csc\left( \frac{7\pi}{4}\right) = \frac{1}{\sin\left( \frac{7\pi}{4}\right)} = \frac{1}{- \sqrt{2}/2} = - \frac{2}{\sqrt{2}} = - \sqrt{2}$.

\item  We have two ways to proceed to determine  $\tan(225^{\circ})$.  First, we can use Theorem \ref{recipquotid} and note that $\tan(225^{\circ})  = \frac{\sin(225^{\circ})}{\cos(225^{\circ})}$.  Since $\sin(225^{\circ}) = \cos(225^{\circ}) = -\frac{\sqrt{2}}{2}$, $\tan(225^{\circ}) = 1$.

\smallskip

Another way to proceed is to note that $225^{\circ}$ has a reference angle of $45^{\circ}$.   Per Theorem \ref{genrefanglethm},  $\tan(225^{\circ}) = \pm \tan(45^{\circ}) = \pm 1$.  Since  $225^{\circ}$ is a Quadrant III angle, where both the $x$ and $y$ coordinates of points are both negative, and tangent is defined as the \textit{ratio} of coordinates $\frac{y}{x}$, we know $\tan(225^{\circ}) > 0$.  Hence,  $\tan(225^{\circ}) = 1$.

\item  As with the previous example, we have two ways to proceed.  Using Theorem \ref{recipquotid}, we have $\cot\left(-\frac{7 \pi}{6} \right) = \frac{\cos\left(-\frac{7 \pi}{6} \right)}{\sin\left(-\frac{7 \pi}{6} \right)}$.  Since $\cos\left(-\frac{7 \pi}{6} \right) = -\frac{\sqrt{3}}{2}$ and $\sin\left(-\frac{7 \pi}{6} \right) = \frac{1}{2}$, we get  $\cot\left(-\frac{7 \pi}{6} \right) = - \sqrt{3}$.

\smallskip

Alternatively, we note $-\frac{7 \pi}{6}$ is a Quadrant II angle with reference angle $\frac{\pi}{6}$.  Hence, Theorem \ref{genrefanglethm} tells us $\cot\left(-\frac{7 \pi}{6} \right) = \pm \cot\left(\frac{\pi}{6} \right) = \pm \sqrt{3}$.  Since $-\frac{7 \pi}{6}$ is a Quadrant II angle, where the $x$ and $y$ coordinates have different signs, and cotangent is defined as the ratio of coordinates $\frac{x}{y}$, we know $\cot\left(-\frac{7 \pi}{6} \right)<0$.  Hence, $\cot\left(-\frac{7 \pi}{6} \right) = -\sqrt{3}$.

\end{enumerate}

\item  \begin{enumerate}

\item  To solve $\sec(\theta) = 2$, we convert to cosines and get $\frac{1}{\cos(\theta)} = 2$ or $\cos(\theta) = \frac{1}{2}$.  This is the exact same equation we solved in Example \ref{solveforangle}, number \ref{cosineishalf}, so we know the answer is:  $\theta = \frac{\pi}{3} + 2\pi k$ or $\theta = \frac{5\pi}{3} + 2\pi k$ for integers $k$.

\item From the table of common values, we see  $\tan\left(\frac{\pi}{3}\right) = \sqrt{3}$.  According to Theorem \ref{genrefanglethm}, we  know  the solutions to $\tan(\theta) = \sqrt{3}$ must, therefore, have a reference angle of $\frac{\pi}{3}$. 

\smallskip

To find the quadrants in which our solutions lie, we note that tangent is defined as the ratio $\frac{y}{x}$ of points $(x,y)$ on the Unit Circle.  Hence, tangent is positive when $x$ and $y$ have the same sign (i.e., when they are both positive or both negative.) This happens in Quadrants I and III. 

\smallskip

 In Quadrant I, we get the solutions: $\theta = \frac{\pi}{3} + 2\pi k$ for integers $k$, and for Quadrant III, we get $\theta = \frac{4\pi}{3} + 2\pi k$ for integers $k$.  While these descriptions of the solutions are correct, they can be combined into one list as $\theta = \frac{\pi}{3} + \pi k$ for integers $k$. The latter form of the solution is best understood looking at the geometry of the situation in the diagram below.\footnote{See Example \ref{solveforangle} number \ref{cosineiszero} in Section \ref{TheCircularFunctionsSineandCosine} for another example of this kind of simplification of the solution.}
 

\begin{tabular}{cc}

\begin{mfpic}[15]{-5.25}{5.25}{-5.25}{5.5}
\axes
\tlabel(5.25,-0.5){\scriptsize $x$}
\tlabel(0.25,5.25){\scriptsize $y$}
\tlabel(4.6,-1){\scriptsize $1$}
\tlabel(0.25,4.6){\scriptsize $1$}
\xmarks{-4.5, 4.5}
\ymarks{-4.5 step 4.5 until 4.5}
\drawcolor[gray]{0.7}
\circle{(0,0),4.5}
\drawcolor{black}
\arrow \reverse \arrow \parafcn{5, 55, 5}{1.5*dir(t)}
\tlabel(1.5, 1){$\frac{\pi}{3}$}
\point[4pt]{(0,0), (2.25, 3.8971)}
\penwd{1.25pt}
\arrow \polyline{(5.25, 0), (0,0), (2.5, 4.330)}
\end{mfpic} 

&

\hspace{.75in}

\begin{mfpic}[15]{-5.25}{5.25}{-5.25}{5.5}
\axes
\tlabel(5.25,-0.5){\scriptsize $x$}
\tlabel(0.25,5.25){\scriptsize $y$}
\tlabel(4.6,-1){\scriptsize $1$}
\tlabel(0.25,4.6){\scriptsize $1$}
\xmarks{-4.5, 4.5}
\ymarks{-4.5 step 4.5 until 4.5}
\drawcolor[gray]{0.7}
\circle{(0,0),4.5}
\drawcolor{black}

\arrow \reverse \arrow \parafcn{185, 235, 5}{2*dir(t)}
\tlabel(-2.6, -1.5){$\frac{\pi}{3}$}
\point[4pt]{(0,0), (-2.25, -3.8971)}
\arrow \dashed \polyline{(0,0), (2.5, 4.330)}
\arrow \reverse \arrow \parafcn{5, 55, 5}{1.5*dir(t)}
\tlabel(1.5, 1){$\frac{\pi}{3}$}
\tlabel(-1.5, 2){$\pi$}
\point[4pt]{(2.25, 3.8971)}
\arrow \reverse \arrow \parafcn{65, 235, 5}{1.5*dir(t)}
\penwd{1.25pt}
\arrow \polyline{(5.25, 0), (0,0), (-2.5, -4.330)}
\end{mfpic} 
\end{tabular}
  

\item  From the table of common values, we see that $\frac{\pi}{4}$ has a cotangent of $1$, which means the solutions to $\cot(\theta) = -1$ have a reference angle of $\frac{\pi}{4}$. 

\smallskip

To find the quadrants in which our solutions lie, we note that $\cot(\theta) = \frac{x}{y}$ for a point $(x,y)$ on the Unit Circle where $y \neq 0$. If $\cot(\theta)$ is negative, then $x$ and $y$ must have different signs (i.e., one positive and one negative.)  Hence, our solutions lie in Quadrants II and IV.  

\smallskip

Our Quadrant II solution is $\theta = \frac{3\pi}{4} + 2\pi k$, and for Quadrant IV, we get $\theta = \frac{7\pi}{4} + 2\pi k$ for integers $k$.  As in the previous problem, we can combine these solutions as:  $\theta = \frac{3\pi}{4} + \pi k$ for integers $k$.


\begin{tabular}{cc}

\begin{mfpic}[15]{-5.25}{5.25}{-5.25}{5.5}
\axes
\tlabel(5.25,-0.5){\scriptsize $x$}
\tlabel(0.25,5.25){\scriptsize $y$}
\tlabel(4.6,-1){\scriptsize $1$}
\tlabel(0.25,4.6){\scriptsize $1$}
\xmarks{-4.5, 4.5}
\ymarks{-4.5 step 4.5 until 4.5}
\drawcolor[gray]{0.7}
\circle{(0,0),4.5}
\drawcolor{black}
\arrow \reverse \arrow \parafcn{140, 175, 5}{1.5*dir(t)}
\tlabel[cc](-2, 1){$\frac{\pi}{4}$}
\point[4pt]{(0,0), (-3.1820, 3.1820)}
\penwd{1.25pt}
\arrow \polyline{(5.25,0), (0,0), (-3.5355, 3.5355)}
\end{mfpic} 

&

\hspace{.75in}

\begin{mfpic}[15]{-5.25}{5.25}{-5.25}{5.5}
\axes
\tlabel(5.25,-0.5){\scriptsize $x$}
\tlabel(0.25,5.25){\scriptsize $y$}
\tlabel(4.6,-1){\scriptsize $1$}
\tlabel(0.25,4.6){\scriptsize $1$}
\xmarks{-4.5, 4.5}
\ymarks{-4.5 step 4.5 until 4.5}
\drawcolor[gray]{0.7}
\circle{(0,0),4.5}
\drawcolor{black}
\arrow \reverse \arrow \parafcn{320, 355, 5}{1.5*dir(t)}
\tlabel[cc](2, -1){$\frac{\pi}{4}$}
\tlabel[cc](-1.77, -1.77){$\pi$}
\point[4pt]{(0,0), (3.1820, -3.1820), (-3.1820, 3.1820)}
\arrow \dashed \polyline{(0,0), (-3.5355, 3.5355)}
\arrow \reverse \arrow \parafcn{140, 175, 5}{2*dir(t)}
\arrow \reverse \arrow \parafcn{140, 310, 5}{1.5*dir(t)}
\tlabel[cc](-2.31, 0.96){$\frac{\pi}{4}$}
\penwd{1.25pt}
\arrow \polyline{(5.25,0), (0,0), (3.5355, -3.5355)}
\end{mfpic}
\end{tabular}

\end{enumerate}
\end{enumerate}

\vspace{-.3in} \qed

\end{example}

A few remarks about Example \ref{circularfunctionsex} are in order.  First note that the signs ($\pm$) of secant and cosecant are the same as the signs of cosine and sine, respectively.  

\smallskip

On the other hand, since tangent and cotangent are defined in terms of the \textit{ratios} of coordinates $x$ and $y$, tangent and cotangent are positive in Quadrants I and III (where both $x$ and $y$ have the same sign) and negative in Quadrants II and IV (where $x$ and $y$ have opposite signs.)  

\smallskip
 
 The diagram below on the left summarizes which circular functions are positive in which quadrants.
 
 \smallskip
 
 \begin{multicols}{2}
 
\begin{mfpic}[18]{-4}{5}{-5}{5}
\axes
\tlabel(4.75,-0.5){\scriptsize $x$}
\tlabel(0.25,5){\scriptsize $y$}
\tlabel(3.1,-0.75){\scriptsize $(1,0)$}
\tlabel(-4.5,-0.75){\scriptsize $(-1,0)$}
\tlabel(0.25,3.1){ \scriptsize $(0,1)$}
\tlabel(0.25,-3.4){\scriptsize $(0,-1)$}
\tlabel[cc](1,0.75){\scriptsize All}
\tlabel[cc](-1.5,0.75){\scriptsize $\sin(\theta)$, $\csc(\theta)$}
\tlabel[cc](-1.5,-0.75){\scriptsize $\tan(\theta)$, $\cot(\theta)$}
\tlabel[cc](1.5,-0.75){\scriptsize $\cos(\theta)$, $\sec(\theta)$}
\tcaption{Postive Circular Functions}
\tlabel[cc](-2.83,-2.83){$(-,-)$}
\tlabel[cc](2.83,2.83){$(+, +)$}
\tlabel[cc](-2.83,2.83){$(-,+)$}
\tlabel[cc](2.83,-2.83){$(+,-)$}
\xmarks{-3 step 3 until 3}
\ymarks{-3 step 3 until 3}
\drawcolor[gray]{0.7}
\circle{(0,0),3}
\point[4pt]{(-2.1213, -2.1213),(2.1213, 2.1213), (2.1213, -2.1213),(-2.1213, 2.1213), (0,3), (3,0), (0,-3), (-3,0) }
\end{mfpic} 

\begin{mfpic}[18]{-4}{5}{-5}{5}
\axes
\tlabel(4.75,-0.5){\scriptsize $x$}
\tlabel(0.25,5){\scriptsize $y$}
\tlabel(3.1,-0.75){\scriptsize $(1,0)$}
\tlabel(-4.5,-0.75){\scriptsize $(-1,0)$}
\tlabel(0.25,3.1){ \scriptsize $(0,1)$}
\tlabel(0.25,-3.4){\scriptsize $(0,-1)$}
\tcaption{The period of $\tan(\theta)$ and $\cot(\theta)$ is $\pi$}
\dashed \polyline{(-2.25, -3.9), (2.25, 3.9)}
\arrow \reverse \arrow \parafcn{65, 235, 5}{2.25*dir(t)}
\arrow \reverse \arrow \parafcn{245, 415, 5}{2.25*dir(t)}
\arrow \reverse \arrow \parafcn{5, 55, 5}{1.25*dir(t)}
\arrow \reverse \arrow \parafcn{185, 235, 5}{1.25*dir(t)}
\tlabel[cc](-1.8, 1.8){$\pi$}
\tlabel[cc](1.8, -1.8){$\pi$}
\xmarks{-3 step 3 until 3}
\ymarks{-3 step 3 until 3}
\drawcolor[gray]{0.7}
\circle{(0,0),3}
\point[4pt]{(-1.5, -2.6),(1.5, 2.6) }
\end{mfpic} 


\end{multicols}

Also note it is no coincidence that both of our solutions to the equations involving tangent and cotangent in  Example \ref{circularfunctionsex} could be simplified to just one list of angles differing by multiples of $\pi$.  

\smallskip

Indeed, any two angles that are $\pi$ units apart will not only have the same reference angle, but points on their terminal sides on the Unit Circle will be reflections through the origin, as illustrated above on the right.

\smallskip

It follows that the tangent and cotangent of such angles (if defined) will be the same, which means the period of these function is (at most) $\pi$.  

\smallskip

Using an argument similar to the one we used to establish the period of sine and cosine in Section \ref{GraphsofSineandCosine}, we note that if $\tan(x+p) = \tan(x)$ for all real numbers $x$, then, in particular,   $\tan(p) = \tan(0+p) = \tan(0) = 0$.  Hence,  $p$ is a multiple of $\pi$, and the smallest multiple of $\pi$ is $\pi$ itself.  

\smallskip

Hence, the period of tangent (and cotangent) is $\pi$, and we will see the consequences of this both when solving equations in this section and when graphing these functions in Section \ref{GraphsofOtherCircularFunctions}.

\smallskip  

As with sine and cosine, the circular functions defined in Definition \ref{circularfunctions}  agree with those put forth in Definitions \ref{righttrianglesinecosinetangent} and \ref{righttriangletherest} in Section \ref{AppRightTrig} for acute angles situated in right triangles.  The argument is identical to the one given in Section \ref{TheCircularFunctionsSineandCosine} and is left to the reader.  

\smallskip

Moreover,  Definition \ref{circularfunctions}  can be extended to circles of arbitrary radius $r>0$ using the same similarity arguments in Section \ref{cosinesinebeyond} to generalize Definition \ref{sinecosineunitcircledefn} to Theorem \ref{cosinesinecircle} as summarized below.

\smallskip

%% \colorbox{ResultColor}{\bbm

\begin{theorem} \label{circularfunctionscircle} Suppose $Q(x,y)$ is the point on the terminal side of an angle $\theta$ (plotted in standard position) which lies on the circle of radius $r$,  $x^2+y^2 = r^2$. Then:

\begin{itemize}

\item  $\sin(\theta) = \dfrac{y}{r} = \dfrac{y}{\sqrt{x^2+y^2}}$ \index{sine ! of an angle}

\item $\cos(\theta)= \dfrac{x}{r}  = \dfrac{x}{\sqrt{x^2+y^2}}$ \index{cosine ! of an angle}

\item $\tan(\theta) = \dfrac{y}{x}$, provided $x \neq 0$. \index{tangent ! of an angle}

\item $\sec(\theta) = \dfrac{r}{x} = \dfrac{\sqrt{x^2+y^2}}{x}$, provided $x \neq 0$. \index{secant ! of an angle}

\item $\csc(\theta) = \dfrac{r}{y} = \dfrac{\sqrt{x^2+y^2}}{y}$, provided $y \neq 0$. \index{cosecant ! of an angle}

\item $\cot(\theta) = \dfrac{x}{y}$, provided $y \neq 0$. \index{cotangent ! of an angle}

\end{itemize}

\end{theorem}

%% \ebm}

\smallskip

We make good use of Theorem \ref{circularfunctionscircle}  in the following example.

\begin{example} \label{circularfunctionscircleex}  Use Theorem \ref{circularfunctionscircle} to solve the following.

\begin{enumerate}

\item  Suppose the terminal side of $\theta$, when plotted in standard position, contains the point $Q(3,4)$.  Find the values of the six circular functions of $\theta$.

\item  Suppose $\theta$ is a Quadrant IV angle with $\cot(\theta) = -4$.  Find the values of the five remaining circular functions of $\theta$.

\item Find $\sin\left(\theta\right)$, where $\sec(\theta) = -\sqrt{5}$ and $\theta$ is a Quadrant II angle.

\item \label{commontanmistake} Find $\cos\left(\theta\right)$, where $\tan(\theta) = 3$ and $\pi < \theta < \frac{3\pi}{2}$.

\end{enumerate}

{\bf Solution.}

\begin{enumerate}

\item    Since $x = 3$ and $y=4$, from $x^2+y^2 = r^2$, $(3)^2+(4)^2 = r^2$ so $r^2 = 25$, or $r = 5$. Theorem \ref{circularfunctionscircle} tells us $\sin(\theta) = \frac{4}{5}$, $\cos(\theta) = \frac{3}{5}$,  $\tan(\theta) = \frac{4}{3}$, $\sec(\theta) = \frac{5}{3}$, $\csc(\theta) = \frac{5}{4}$,  and $\cot(\theta) = \frac{3}{4}$.

\item In order to use Theorem \ref{circularfunctionscircle}, we need to find a point $Q(x,y)$ which lies on the terminal side of $\theta$, when $\theta$ is plotted in standard position.  


\smallskip

We have that $\cot(\theta) = -4 =  \frac{x}{y}$.  Since  $\theta$ is a Quadrant IV angle, we also know $x>0$ and $y< 0$.  Rewriting   $-4 = \frac{4}{-1}$, we choose\footnote{We could have just as easily chosen $x=8$ and $y=-2$ - just so long as $x>0$, $y<0$ and $\frac{x}{y} = -4$.}   $x = 4$ and $y = -1$ so that $r = \sqrt{x^2+y^2} = \sqrt{(4)^2 + (-1)^2} = \sqrt{17}$.  

\smallskip

Applying Theorem \ref{circularfunctionscircle}, we find  $\sin(\theta) =- \frac{1}{\sqrt{17}} = -\frac{\sqrt{17}}{17}$, $\cos(\theta) = \frac{4}{\sqrt{17}} = \frac{4 \sqrt{17}}{17}$,  $\tan(\theta) = -\frac{1}{4}$, $\sec(\theta) = \frac{\sqrt{17}}{4}$, and $\csc(\theta) = - \sqrt{17}$.

\begin{tabular}{cc}

\begin{mfpic}[18]{-5}{5}{-4}{4.75}
\axes
\tlabel(4.75,-0.5){\scriptsize $x$}
\tlabel(0.25,4.5){\scriptsize $y$}
\tlabel(2.75,3){\scriptsize $Q(3,4)$}
\tlabel[cc](2,1.5){\scriptsize $r=5$}
\tcaption{$Q(3,4)$ lies on a circle of radius $5$ units,}
\xmarks{-2.68, -2,  -1.34, -0.67, 0.67, 1.34, 2, 2.68}
\ymarks{-2.68, -2,  -1.34, -0.67, 0.67, 1.34, 2, 2.68}
\drawcolor[gray]{0.7}
\circle{(0,0),3.354}
\drawcolor{black}
\penwd{1.25pt}
\arrow \reverse \arrow \polyline{(5,0),(0,0), (3, 4)}
\point[4pt]{(0,0), (2.012, 2.683)}
\tlpointsep{4pt}
\scriptsize 
\axislabels {x}{ {$-3 \hspace{7pt}$} -2, {$-1 \hspace{7pt}$} -0.67,  {$1$} 0.67,  {$3$} 2, {$-2 \hspace{7pt}$} -1.34, {$-4 \hspace{7pt}$} -2.68, {$2$} 1.34, {$4$} 2.68}
\axislabels {y}{ {$-2$} -1.34, {$-4$} -2.68, {$2$} 1.34, {$4$} 2.68,  {$1$} 0.67,  {$3$} 2,  {$-1$} -0.67,  {$-3$} -2 }
\normalsize
\end{mfpic}

&

\begin{mfpic}[18]{-5}{5}{-4}{4.75}
\axes
\tlabel(4.75,-0.5){\scriptsize $x$}
\tlabel(0.25,4.5){\scriptsize $y$}
\tlabel[cc](1.4, -1.25){\scriptsize $r = \sqrt{17}$}
\tlabel[cc](4, -1.5){\scriptsize $(4, -1)$}
\tcaption{$\theta$ is Quadrant IV with $\cot(\theta) = -4$.}
\xmarks{-3.24, -1.62, 1.62, 3.24, 0.81, -0.81, 2.43, -2.43}
\ymarks{-3.24, -1.62, 1.62, 3.24, 0.81, -0.81, 2.43, -2.43}
\drawcolor[gray]{0.7}
\circle{(0,0),3.354}
\drawcolor{black}
\penwd{1.25pt}
\arrow \reverse \arrow \polyline{(5,0),(0,0), (4.92, -1.23)}
\point[4pt]{(0,0), (3.24, -0.81)}
\tlpointsep{4pt}
\scriptsize 
\axislabels {x}{ {$4$} 3.24, {$3$} 2.43, {$-4 \hspace{7pt}$} -3.24, {$-2$ \hspace{7pt}} -1.62, {$-1 \hspace{7pt}$} -0.81, {$-3 \hspace{7pt}$} -2.43}
\axislabels {y}{ {$4$} 3.24, {$2$} 1.62, {$1$} 0.81, {$3$} 2.43, {$-4$} -3.24, {$-2$} -1.62, {$-1$} -0.81, {$-3$} -2.43}
\normalsize
\end{mfpic}


\end{tabular}


\item To find $\sin(\theta)$ using Theorem \ref{circularfunctionscircle}, we need to determine the $y$-coordinate of a point $Q(x,y)$ on the terminal side of $\theta$, when $\theta$ is plotted in standard position,  and the corresponding radius $r$. 

\smallskip

Since   $\sec(\theta) = \frac{r}{x}$ and $r > 0$, we rewrite  $\sec(\theta) = \frac{r}{x}  = -\sqrt{5} = \frac{\sqrt{5}}{-1}$ and take $r = \sqrt{5}$ and $x = -1$.

\smallskip

To find $y$, we substitute $x=-1$ and $r=\sqrt{5}$ into $x^2+y^2 = r^2$ to get $(-1)^2+y^2=(\sqrt{5})^2$.  We find $y^2 = 4$ or $y = \pm 2$.   Since $\theta$ is a Quadrant II angle, we select $y = 2$.

\smallskip

Hence, $\sin(\theta) = \frac{y}{r} = \frac{2}{\sqrt{5}} = \frac{2 \sqrt{5}}{5}$.


\item  We are told $\tan(\theta) = 3$ and  $\pi < \theta < \frac{3\pi}{2}$, so we know $\theta$ is a Quadrant III angle. 

\smallskip

 To find $\cos(\theta)$ using Theorem \ref{circularfunctionscircle},  we need to find the $x$-coordinate of a point $Q(x,y)$ on the terminal side of $\theta$, when $\theta$ is plotted in standard position,  and the corresponding radius, $r$.

\smallskip

Since $\tan(\theta) = \frac{y}{x}$ and $\theta$ is a Quadrant III angle, we rewrite $\tan(\theta) = 3 = \frac{-3}{-1}  = \frac{y}{x}$ and choose $x = -1$ and $y = -3$.  From $x^2+y^2 = r^2$, we get $r = \sqrt{10}$.

\smallskip

Hence, $\cos(\theta) = \frac{x}{r} = \frac{-1}{\sqrt{10}} = -\frac{\sqrt{10}}{10}$.

\begin{tabular}{cc}

\begin{mfpic}[18]{-5}{5}{-4}{4.75}
\axes
\tlabel(4.75,-0.5){\scriptsize $x$}
\tlabel(0.25,4.5){\scriptsize $y$}
\tlabel[cc](-3, 3){\scriptsize $Q(-2, 1)$}
\tlabel[cc](-1.75,1.5){\scriptsize $r=\sqrt{5}$}
\tcaption{$\theta$ is Quadrant II with $\sec(\theta) = -\sqrt{5}$}
\xmarks{-1.5, -3, -4.5, 1.5, 3, 4.5}
\ymarks{-1.5, -3,1.5, 3, 4.5}
\drawcolor[gray]{0.7}
\circle{(0,0),3.354}
\drawcolor{black}
\penwd{1.25pt}
\arrow \reverse \arrow \polyline{(5,0),(0,0), (-2.25, 4.5)}
\point[4pt]{(0,0), (-1.5,3)}
\tlpointsep{4pt}
\scriptsize 
\axislabels {x}{ {$-1 \hspace{7pt}$} -1.5, {$-2 \hspace{7pt}$} -3, {$-3 \hspace{7pt}$} -4.5, {$1$} 1.5, {$2$} 3, {$3$} 4.5}
\axislabels {y}{ {$-2$} -3, {$-1$} -1.5, {$1$} 1.5, {$2$} 3}
\normalsize
\end{mfpic}

&

\begin{mfpic}[18]{-5}{5}{-4}{4.75}
\axes
\tlabel(4.75,-0.5){\scriptsize $x$}
\tlabel(0.25,4.5){\scriptsize $y$}
\tlabel[cc](-2, -1.25){\scriptsize $r = \sqrt{10}$}
\tlabel[cc](-2.5, -3.18){\scriptsize $(-1, -3)$}
\tcaption{$\theta$ is Quadrant III with $\tan(\theta) = 3$.}
\xmarks{1.06, 2.12, 3.18, 4.24, -1.06, -2.12, -3.18, -4.24 }
\ymarks{1.06, 2.12, 3.18, -1.06, -2.12, -3.18 }
\drawcolor[gray]{0.7}
\circle{(0,0),3.354}
\drawcolor{black}
\penwd{1.25pt}
\arrow \reverse \arrow \polyline{(5,0),(0,0), (-1.58, -4.74)}
\point[4pt]{(0,0), (-1.06, -3.18)}
\tlpointsep{4pt}
\scriptsize 
\axislabels {x}{  {$4$} 4.24, {$3$} 3.18, {$2$} 2.12,  {$1$} 1.06, {$-4 \hspace{7pt}$} -4.24, {$-3 \hspace{7pt}$} -3.18, {$-2 \hspace{7pt}$} -2.12,  {$-1 \hspace{7pt}$} -1.06}
\axislabels {y}{ {$3$} 3.18, {$2$} 2.12,  {$1$} 1.06, {$-3$} -3.18, {$-2$} -2.12}
\normalsize
\end{mfpic}


\end{tabular}


\qed


\end{enumerate}

\end{example}



As we did in Section \ref{cosinesinebeyond}, we may consider $\tan(t)$, $\sec(t)$, $\csc(t)$, and $\cot(t)$ as functions \textit{real numbers} by associating each real number $t$ with an angle $\theta$ measuring $t$ radians as discussed on page \pageref{cosinesineequationsrealnumbers} and using Definition \ref{circularfunctions}, or, more generally, Theorem \ref{circularfunctionscircle}.
\smallskip

Alternatively, we could define each of these four functions in terms of $f(t) = \sin(t)$ and $g(t) = \cos(t)$ as demonstrated in Theorem \ref{recipquotid}.  For example, we could simply \textit{define} $\sec(t) = \frac{1}{\cos(t)}$ so long as $\cos(t) \neq 0$.  

\smallskip

Either way, we have the means to explore these functions in  greater  detail.  Before doing so, we'll need practice with these additional four circular functions courtesy of the Exercises.


\newpage

\subsection{Exercises}

%% SKIPPED %% \documentclass{ximera}

\begin{document}
	\author{Stitz-Zeager}
	\xmtitle{TITLE}
\mfpicnumber{1} \opengraphsfile{ExercisesforTheOtherCircularFunctions} % mfpic settings added 


In Exercises \ref{circvaluefirst} - \ref{circvaluelast}, find the exact value or state that it is undefined.

\begin{multicols}{4}

\begin{enumerate}

\item $\tan \left( \dfrac{\pi}{4} \right)$ \vphantom{$\csc \left( \dfrac{5\pi}{6} \right)$} \label{circvaluefirst}
\item $\sec \left( \dfrac{\pi}{6} \right)$ \vphantom{$\csc \left( \dfrac{5\pi}{6} \right)$}
\item $\csc \left( \dfrac{5\pi}{6} \right)$
\item $\cot \left( \dfrac{4\pi}{3} \right)$

\setcounter{HW}{\value{enumi}}

\end{enumerate}

\end{multicols}

\begin{multicols}{4}

\begin{enumerate}

\setcounter{enumi}{\value{HW}}

\item $\tan \left( -\dfrac{11\pi}{6} \right)$
\item $\sec \left( -\dfrac{3\pi}{2} \right)$
\item $\csc \left( -\dfrac{\pi}{3} \right)$ \vphantom{$\csc \left( \dfrac{5\pi}{6} \right)$}
\item $\cot \left( \dfrac{13\pi}{2} \right)$

\setcounter{HW}{\value{enumi}}

\end{enumerate}

\end{multicols}

\begin{multicols}{4}

\begin{enumerate}

\setcounter{enumi}{\value{HW}}

\item $\tan \left( 117\pi \right)$ \vphantom{$\csc \left( \dfrac{5\pi}{6} \right)$}
\item $\sec \left( -\dfrac{5\pi}{3} \right)$
\item $\csc \left( 3\pi \right)$ \vphantom{$\csc \left( \dfrac{5\pi}{6} \right)$}
\item $\cot \left( -5\pi \right)$ \vphantom{$\csc \left( \dfrac{5\pi}{6} \right)$}

\setcounter{HW}{\value{enumi}}

\end{enumerate}

\end{multicols}

\begin{multicols}{4}

\begin{enumerate}

\setcounter{enumi}{\value{HW}}

\item $\tan \left( \dfrac{31\pi}{2} \right)$
\item $\sec \left( \dfrac{\pi}{4} \right)$ \vphantom{$\csc \left( \dfrac{5\pi}{6} \right)$}
\item $\csc \left( -\dfrac{7\pi}{4} \right)$
\item $\cot \left( \dfrac{7\pi}{6} \right)$

\setcounter{HW}{\value{enumi}}

\end{enumerate}

\end{multicols}

\begin{multicols}{4}

\begin{enumerate}

\setcounter{enumi}{\value{HW}}

\item $\tan \left( \dfrac{2\pi}{3} \right)$
\item $\sec \left( -7\pi \right)$ \vphantom{$\csc \left( \dfrac{5\pi}{6} \right)$}
\item $\csc \left( \dfrac{\pi}{2} \right)$ \vphantom{$\csc \left( \dfrac{5\pi}{6} \right)$}
\item $\cot \left( \dfrac{3\pi}{4} \right)$ \label{circvaluelast}

\setcounter{HW}{\value{enumi}}

\end{enumerate}

\end{multicols}

In Exercises \ref{whereisanglefirst} - \ref{whereisanglelast}, use the given the information to determine the quadrant in which the terminal side of the angle lies when plotted in standard position.

\begin{multicols}{2}
\begin{enumerate}
\setcounter{enumi}{\value{HW}}

\item  \label{whereisanglefirst} $\sin(\theta) > 0$ but $\tan(\theta) < 0$.

\item  $\cot(\alpha) > 0$ but $\cos(\alpha) < 0$.

\setcounter{HW}{\value{enumi}}
\end{enumerate}
\end{multicols}

\begin{multicols}{2}
\begin{enumerate}
\setcounter{enumi}{\value{HW}}

\item  $\sin(\beta) > 0$ and $\tan(\beta) > 0$.

\item  \label{whereisanglelast} $\cos(\gamma) > 0$ but $\cot(\gamma) < 0$.

\setcounter{HW}{\value{enumi}}
\end{enumerate}
\end{multicols}


In Exercises \ref{findothercircfirst} - \ref{findothercirclast}, use the given the information to find the exact values of the circular functions of $\theta$.

\begin{multicols}{2}

\begin{enumerate}

\setcounter{enumi}{\value{HW}}

\item $\sin(\theta) = \dfrac{3}{5}$ with $\theta$ in Quadrant II \label{findothercircfirst}
\item $\tan(\theta) = \dfrac{12}{5}$ with $\theta$ in Quadrant III

\setcounter{HW}{\value{enumi}}

\end{enumerate}

\end{multicols}

\begin{multicols}{2}

\begin{enumerate}

\setcounter{enumi}{\value{HW}}

\item $\csc(\theta) = \dfrac{25}{24}$ with $\theta$ in Quadrant I
\item $\sec(\theta) = 7$ with $\theta$ in Quadrant IV \vphantom{$\dfrac{25}{24}$}

\setcounter{HW}{\value{enumi}}

\end{enumerate}

\end{multicols}

\begin{multicols}{2}

\begin{enumerate}

\setcounter{enumi}{\value{HW}}

\item $\csc(\theta) = -\dfrac{10\sqrt{91}}{91}$ with $\theta$ in Quadrant III
\item $\cot(\theta) = -23$ with $\theta$ in Quadrant II \vphantom{$\dfrac{10}{\sqrt{91}}$}

\setcounter{HW}{\value{enumi}}

\end{enumerate}

\end{multicols}

\begin{multicols}{2}

\begin{enumerate}

\setcounter{enumi}{\value{HW}}

\item  $\tan(\theta) = -2$ with $\theta$ in Quadrant IV.
\item  $\sec(\theta) = -4$ with $\theta$ in Quadrant II.

\setcounter{HW}{\value{enumi}}

\end{enumerate}

\end{multicols}

\begin{multicols}{2}

\begin{enumerate}

\setcounter{enumi}{\value{HW}}

\item $\cot(\theta) = \sqrt{5}$ with $\theta$ in Quadrant III. \vphantom{$\dfrac{25}{24}$}
\item  $\cos(\theta) = \dfrac{1}{3}$ with $\theta$ in Quadrant I.

\setcounter{HW}{\value{enumi}}

\end{enumerate}

\end{multicols}

\begin{multicols}{2}

\begin{enumerate}

\setcounter{enumi}{\value{HW}}

\item  $\cot(\theta) = 2$ with $0  < \theta < \dfrac{\pi}{2}$.
\item  $\csc(\theta) = 5$ with $\dfrac{\pi}{2} < \theta < \pi$.

\setcounter{HW}{\value{enumi}}

\end{enumerate}

\end{multicols}

\begin{multicols}{2}

\begin{enumerate}

\setcounter{enumi}{\value{HW}}

\item  $\tan(\theta) = \sqrt{10}$ with $\pi < \theta < \dfrac{3\pi}{2}$.
\item  $\sec(\theta) = 2\sqrt{5}$ with $\dfrac{3\pi}{2} < \theta < 2\pi$. \label{findothercirclast}

\setcounter{HW}{\value{enumi}}

\end{enumerate}

\end{multicols}

\pagebreak

In Exercises \ref{circcalcfirst} - \ref{circcalclast}, use your calculator to approximate the given value to three decimal places.  Make sure your calculator is in the proper angle measurement mode!

\begin{multicols}{4}

\begin{enumerate}

\setcounter{enumi}{\value{HW}}

\item $\csc(78.95^{\circ})$ \label{circcalcfirst}
\item $\tan(-2.01)$
\item $\cot(392.994)$
\item $\sec(207^{\circ})$

\setcounter{HW}{\value{enumi}}

\end{enumerate}

\end{multicols}

\begin{multicols}{4}

\begin{enumerate}

\setcounter{enumi}{\value{HW}}

\item $\csc(5.902)$
\item $\tan(39.672^{\circ})$
\item $\cot(3^{\circ})$
\item $\sec(0.45)$ \label{circcalclast}

\setcounter{HW}{\value{enumi}}

\end{enumerate}

\end{multicols}


In Exercises \ref{circequanglefirst} - \ref{circequanglelast}, find all of the angles which satisfy the equation.

\begin{multicols}{4}

\begin{enumerate}

\setcounter{enumi}{\value{HW}}

\item $\tan(\theta) = \sqrt{3}$ \vphantom{$\dfrac{\sqrt{3}}{3}$} \label{circequanglefirst}
\item $\sec(\theta) = 2$ \vphantom{$\dfrac{\sqrt{3}}{3}$}
\item $\csc(\theta) = -1$ \vphantom{$\dfrac{\sqrt{3}}{3}$}
\item $\cot(\theta) = \dfrac{\sqrt{3}}{3}$

\setcounter{HW}{\value{enumi}}

\end{enumerate}

\end{multicols}

\begin{multicols}{4}

\begin{enumerate}

\setcounter{enumi}{\value{HW}}

\item $\tan(\theta) = 0$
\item $\sec(\theta) = 1$
\item $\csc(\theta) = 2$
\item $\cot(\theta) = 0$

\setcounter{HW}{\value{enumi}}

\end{enumerate}

\end{multicols}

\begin{multicols}{4}

\begin{enumerate}

\setcounter{enumi}{\value{HW}}

\item $\tan(\theta) = -1$ \vphantom{$\dfrac{1}{2}$}
\item $\sec(\theta) = 0$ \vphantom{$\dfrac{1}{2}$}
\item $\csc(\theta) = -\dfrac{1}{2}$
\item  $\sec(\theta) = -1$ \vphantom{$\dfrac{1}{2}$}

\setcounter{HW}{\value{enumi}}

\end{enumerate}

\end{multicols}

\begin{multicols}{4}

\begin{enumerate}

\setcounter{enumi}{\value{HW}}

\item  $\tan(\theta) = -\sqrt{3}$
\item  $\csc(\theta) = -2$ \vphantom{$\sqrt{3}$}
\item  $\cot(\theta) = -1$ \vphantom{$\sqrt{3}$} \label{circequanglelast}

\setcounter{HW}{\value{enumi}}

\end{enumerate}

\end{multicols}

In Exercises \ref{circequtfirst} - \ref{circequtlast}, solve the equation for $t$.  Give exact values.

\begin{multicols}{4}

\begin{enumerate}

\setcounter{enumi}{\value{HW}}

\item $\cot(t) = 1$ \vphantom{$\dfrac{2\sqrt{3}}{3}$} \label{circequtfirst}
\item  $\tan(t) = \dfrac{\sqrt{3}}{3}$ \vphantom{$\dfrac{2\sqrt{3}}{3}$}
\item $\sec(t) = -\dfrac{2\sqrt{3}}{3}$
\item $\csc(t) = 0$ \vphantom{$\dfrac{2\sqrt{3}}{3}$} 

\setcounter{HW}{\value{enumi}}

\end{enumerate}

\end{multicols}

\begin{multicols}{4}

\begin{enumerate}

\setcounter{enumi}{\value{HW}}

\item $\cot(t) = -\sqrt{3}$ \vphantom{$\dfrac{2\sqrt{3}}{3}$} 
\item $\tan(t) = -\dfrac{\sqrt{3}}{3}$
\item $\sec(t) = \dfrac{2\sqrt{3}}{3}$
\item $\csc(t) = \dfrac{2\sqrt{3}}{3}$ \label{circequtlast}

\setcounter{HW}{\value{enumi}}

\end{enumerate}

\end{multicols}

In Exercises \ref{decomposebasicothercircularfirst} - \ref{decomposebasicothercircularlast}, write the given function as a nontrivial decomposition of functions as directed.

\begin{enumerate}
\setcounter{enumi}{\value{HW}}

\item  For $f(t) = 3t^2 + 2 \tan(3t)$, find functions $g$ and $h$ so that $f=g+h$. \label{decomposebasicothercircularfirst}

\item  For $f(\theta) = \sec(\theta) - \tan(\theta)$, find functions $g$ and $h$ so that $f=g-h$. 

\item  For $f(t) = -\csc(t) \cot(t)$, find functions $g$ and $h$ so that $f=gh$.

\item  For $r(t) = \dfrac{\tan(3t)}{t}$, find functions $f$ and $g$ so $r = \dfrac{f}{g}$.

\item  For $T(\theta) =\tan(4 \theta)$, find functions $f$ and $g$ so $T = g \circ f$.

\item  For $s(\theta) = \sec^{2}(\theta)$, find functions $f$ and $g$ so $s = g \circ f$.

\item  For $L(x) = \ln (\sin(x) )$, find functions $f$ and $g$ so $L = g \circ f$.       

\item  For $\ell(\theta) = \ln | \sec(\theta) - \tan(\theta)|$, find  find functions $f$,  $g$, and $h$ so $\ell = h \circ (f-g)$.\label{decomposebasicothercircularlast}

\item  Let $S(t) = \sin(t)$ and $C(t) = \cos(t)$, $F(t) = \tan(t)$, and $G(t) = \cot(t)$.  Explain why $F = \dfrac{S}{C}$ but $F \neq \dfrac{1}{G}$.

HINT: Think about domains \ldots

\setcounter{HW}{\value{enumi}}
\end{enumerate}


\newpage


\begin{enumerate}
\setcounter{enumi}{\value{HW}}

\item \label{tangentarcexercise}For each function $T(t)$ listed below, compute the average rate of change over the indicated interval.\footnote{See Definition \ref{arc} in Section \ref{AverageRateofChange} for a review of this concept, as needed.}  What trends do you notice? Compare your answer with what you discovered in Section \ref{TheCircularFunctionsSineandCosine} number \ref{sinearcexercise}.  Be sure your calculator is in radian mode!

\[ \begin{array}{|r||c|c|c|}  \hline

 T(t) &  [-0.1, 0.1] & [-0.01, 0.01] &[-0.001, 0.001] \\ \hline
 \tan(t)     &&&  \\  \hline
 \tan(2t)   &&&  \\ \hline
 \tan(3t)   &&&   \\  \hline
 \tan(4t)   &&&   \\  \hline

\end{array} \]

\setcounter{HW}{\value{enumi}}
\end{enumerate}



\begin{enumerate}
\setcounter{enumi}{\value{HW}}


\item\label{sintovertexercise1}  We wish to establish the inequality $\cos(\theta) < \frac{\sin(\theta)}{\theta} < 1$ for $0 < \theta < \frac{\pi}{2}.$  Use the diagram from the beginning of the section, partially reproduced below, to answer the following.

\begin{center}

\begin{mfpic}[20]{-1}{4}{-1}{6}
\axes
\drawcolor[gray]{0.7}
\parafcn{0,90,5}{3*dir(t)}
\drawcolor{black}
\arrow \parafcn{5, 55, 5}{0.75*dir(t)}
\tlabel[cc](0.75,0.5){\scriptsize $\theta$}
\point[3pt]{(0,0), (3,5.196), (3,0)}
\tlabel(3.75,-0.5){\scriptsize $x$}
\tlabel(0.25,5.75){\scriptsize $y$}
\tlabel(0.25,3.1){\scriptsize $1$}
\tlabel(-0.5,-0.5){\scriptsize $O$}
\tlabel(2,-0.5){\scriptsize $B(1,0)$}
\xmarks{0 step 3 until 3}
\ymarks{0 step 3 until 3}
\polyline{(0,0), (3,5.196)}
\polyline{(3,0), (3,5.196)}
\polyline{(2.75,0), (2.75, 0.25), (3,0.25)}
\polyline{(3,0), (1.5, 2.5981)}
\tlabel(1.75,2.6){\scriptsize $P$}
\tlabel(3.25,5.25){\scriptsize $Q$}
\end{mfpic} 

\end{center}

\begin{enumerate}

\item Show that triangle $OPB$ has area $\frac{1}{2} \sin(\theta)$ and triangle $OQB$ has area $\frac{1}{2} \tan(\theta)$.
\item Show that the circular sector $OPB$ with central angle $\theta$ has area $\frac{1}{2} \theta$.
\item Comparing areas, show that $\sin(\theta) < \theta < \tan(\theta)$ for $0 < \theta < \frac{\pi}{2}.$ 
\item Use the inequality $\sin(\theta) < \theta$ to show that $\frac{\sin(\theta)}{\theta} < 1$ for $0 < \theta < \frac{\pi}{2}.$
\item Use the inequality $\theta < \tan(\theta)$ to show that $\cos(\theta) < \frac{\sin(\theta)}{\theta}$ for $0 < \theta < \frac{\pi}{2}.$  Combine this with the previous part to complete the proof.

\end{enumerate}

\item\label{sintovertexercise2} Show that $\cos(\theta) < \frac{\sin(\theta)}{\theta} < 1$ also holds for $-\frac{\pi}{2}< \theta < 0$. 

\item\label{sintovertexercise3}   Use the results from Exercises \ref{sintovertexercise1}  and \ref{sintovertexercise2} along with the Squeeze Theorem,\footnote{Theorem \ref{squeezeth}}  to prove  $\ds{ \lim_{\theta \rightarrow 0}}$ $ \frac{\sin(\theta)}{\theta} = 1$.

\end{enumerate}

\newpage

\subsection{Answers}

\begin{multicols}{3}

\begin{enumerate}

\item $\tan \left( \dfrac{\pi}{4} \right) = 1$ \vphantom{$\dfrac{2\sqrt{3}}{3}$}
\item $\sec \left( \dfrac{\pi}{6} \right) = \dfrac{2\sqrt{3}}{3}$
\item $\csc \left( \dfrac{5\pi}{6} \right) = 2$ \vphantom{$\dfrac{2\sqrt{3}}{3}$}

\setcounter{HW}{\value{enumi}}

\end{enumerate}

\end{multicols}

\begin{multicols}{3}

\begin{enumerate}

\setcounter{enumi}{\value{HW}}

\item $\cot \left( \dfrac{4\pi}{3} \right) = \dfrac{\sqrt{3}}{3}$
\item $\tan \left( -\dfrac{11\pi}{6} \right) = \dfrac{\sqrt{3}}{3}$
\item $\sec \left( -\dfrac{3\pi}{2} \right)$ is undefined 

\setcounter{HW}{\value{enumi}}

\end{enumerate}

\end{multicols}

\begin{multicols}{3}

\begin{enumerate}

\setcounter{enumi}{\value{HW}}

\item $\csc \left( -\dfrac{\pi}{3} \right) = -\dfrac{2\sqrt{3}}{3}$
\item $\cot \left( \dfrac{13\pi}{2} \right) = 0$
\item $\tan \left( 117\pi \right) = 0$ \vphantom{$\dfrac{2\sqrt{3}}{3}$}

\setcounter{HW}{\value{enumi}}

\end{enumerate}

\end{multicols}

\begin{multicols}{3}

\begin{enumerate}

\setcounter{enumi}{\value{HW}}

\item $\sec \left( -\dfrac{5\pi}{3} \right) = 2$
\item $\csc \left( 3\pi \right)$ is undefined \vphantom{$\left( -\dfrac{5\pi}{3} \right)$}
\item $\cot \left( -5\pi \right)$ is undefined \vphantom{$\left( -\dfrac{5\pi}{3} \right)$}

\setcounter{HW}{\value{enumi}}

\end{enumerate}

\end{multicols}

\begin{multicols}{3}

\begin{enumerate}

\setcounter{enumi}{\value{HW}}

\item $\tan \left( \dfrac{31\pi}{2} \right)$ is undefined
\item $\sec \left( \dfrac{\pi}{4} \right) = \sqrt{2}$ \vphantom{$\left( -\dfrac{5\pi}{3} \right)$}
\item $\csc \left( -\dfrac{7\pi}{4} \right) = \sqrt{2}$

\setcounter{HW}{\value{enumi}}

\end{enumerate}

\end{multicols}

\begin{multicols}{3}

\begin{enumerate}

\setcounter{enumi}{\value{HW}}

\item $\cot \left( \dfrac{7\pi}{6} \right) = \sqrt{3}$
\item $\tan \left( \dfrac{2\pi}{3} \right) = -\sqrt{3}$
\item $\sec \left( -7\pi \right) = -1$ \vphantom{$\left( -\dfrac{5\pi}{3} \right)$}

\setcounter{HW}{\value{enumi}}

\end{enumerate}

\end{multicols}

\begin{multicols}{3}

\begin{enumerate}

\setcounter{enumi}{\value{HW}}

\item $\csc \left( \dfrac{\pi}{2} \right) = 1$ \vphantom{$\left( -\dfrac{5\pi}{3} \right)$}
\item $\cot \left( \dfrac{3\pi}{4} \right) = -1$

\setcounter{HW}{\value{enumi}}

\end{enumerate}

\end{multicols}

\begin{multicols}{4}

\begin{enumerate}
\setcounter{enumi}{\value{HW}}

\item   Quadrant II.

\item Quadrant III.

\item  Quadrant I.

\item Quadrant IV.


\setcounter{HW}{\value{enumi}}
\end{enumerate}

\end{multicols}

\begin{enumerate}

\setcounter{enumi}{\value{HW}}

\item $\sin(\theta) = \frac{3}{5}, \cos(\theta) = -\frac{4}{5}, \tan(\theta) = -\frac{3}{4}, \csc(\theta) = \frac{5}{3}, \sec(\theta) = -\frac{5}{4}, \cot(\theta) = -\frac{4}{3}$

\item $\sin(\theta) = -\frac{12}{13}, \cos(\theta) = -\frac{5}{13}, \tan(\theta) = \frac{12}{5}, \csc(\theta) = -\frac{13}{12}, \sec(\theta) = -\frac{13}{5}, \cot(\theta) = \frac{5}{12}$

\item $\sin(\theta) = \frac{24}{25}, \cos(\theta) = \frac{7}{25}, \tan(\theta) = \frac{24}{7}, \csc(\theta) = \frac{25}{24}, \sec(\theta) = \frac{25}{7}, \cot(\theta) = \frac{7}{24}$

\item $\sin(\theta) = \frac{-4\sqrt{3}}{7}, \cos(\theta) = \frac{1}{7}, \tan(\theta) = -4\sqrt{3}, \csc(\theta) = -\frac{7\sqrt{3}}{12}, \sec(\theta) = 7, \cot(\theta) = -\frac{\sqrt{3}}{12}$

\item $\sin(\theta) = -\frac{\sqrt{91}}{10}, \cos(\theta) = -\frac{3}{10}, \tan(\theta) = \frac{\sqrt{91}}{3}, \csc(\theta) = -\frac{10\sqrt{91}}{91}, \sec(\theta) = -\frac{10}{3}, \cot(\theta) = \frac{3\sqrt{91}}{91}$

\item $\sin(\theta) = \frac{\sqrt{530}}{530}, \cos(\theta) = -\frac{23\sqrt{530}}{530}, \tan(\theta) = -\frac{1}{23}, \csc(\theta) = \sqrt{530}, \sec(\theta) = -\frac{\sqrt{530}}{23}, \cot(\theta) = -23$

\item $\sin(\theta) = -\frac{2\sqrt{5}}{5}, \cos(\theta) = \frac{\sqrt{5}}{5}, \tan(\theta) = -2, \csc(\theta) = -\frac{\sqrt{5}}{2}, \sec(\theta) = \sqrt{5}, \cot(\theta) = -\frac{1}{2}$

\item  $\sin(\theta) = \frac{\sqrt{15}}{4}, \cos(\theta) = -\frac{1}{4}, \tan(\theta) = -\sqrt{15}, \csc(\theta) = \frac{4\sqrt{15}}{15}, \sec(\theta) = -4, \cot(\theta) = -\frac{\sqrt{15}}{15}$

\item $\sin(\theta) = -\frac{\sqrt{6}}{6}, \cos(\theta) = -\frac{\sqrt{30}}{6}, \tan(\theta) = \frac{\sqrt{5}}{5}, \csc(\theta) = -\sqrt{6}, \sec(\theta) = -\frac{\sqrt{30}}{5}, \cot(\theta) = \sqrt{5}$

\item $\sin(\theta) = \frac{2\sqrt{2}}{3}, \cos(\theta) = \frac{1}{3}, \tan(\theta) = 2\sqrt{2}, \csc(\theta) = \frac{3\sqrt{2}}{4}, \sec(\theta) = 3, \cot(\theta) = \frac{\sqrt{2}}{4}$

\item $\sin(\theta) = \frac{\sqrt{5}}{5}, \cos(\theta) = \frac{2\sqrt{5}}{5}, \tan(\theta) = \frac{1}{2}, \csc(\theta) = \sqrt{5}, \sec(\theta) = \frac{\sqrt{5}}{2}, \cot(\theta) = 2$

\item $\sin(\theta) = \frac{1}{5}, \cos(\theta) = -\frac{2\sqrt{6}}{5}, \tan(\theta) = -\frac{\sqrt{6}}{12}, \csc(\theta) = 5, \sec(\theta) = -\frac{5\sqrt{6}}{12}, \cot(\theta) = -2\sqrt{6}$

\item $\sin(\theta) = -\frac{\sqrt{110}}{11}, \cos(\theta) = -\frac{\sqrt{11}}{11}, \tan(\theta) = \sqrt{10}, \csc(\theta) = -\frac{\sqrt{110}}{10}, \sec(\theta) = -\sqrt{11}, \cot(\theta) = \frac{\sqrt{10}}{10}$

\item $\sin(\theta) = -\frac{\sqrt{95}}{10}, \cos(\theta) = \frac{\sqrt{5}}{10}, \tan(\theta) = -\sqrt{19}, \csc(\theta) = -\frac{2\sqrt{95}}{19}, \sec(\theta) = 2\sqrt{5}, \cot(\theta) = -\frac{\sqrt{19}}{19}$

\setcounter{HW}{\value{enumi}}

\end{enumerate}

\begin{multicols}{2}

\begin{enumerate}

\setcounter{enumi}{\value{HW}}

\item $\csc(78.95^{\circ}) \approx 1.019$
\item $\tan(-2.01) \approx 2.129$

\setcounter{HW}{\value{enumi}}

\end{enumerate}

\end{multicols}

\begin{multicols}{2}

\begin{enumerate}

\setcounter{enumi}{\value{HW}}

\item $\cot(392.994) \approx 3.292$
\item $\sec(207^{\circ}) \approx -1.122$

\setcounter{HW}{\value{enumi}}

\end{enumerate}

\end{multicols}

\begin{multicols}{2}

\begin{enumerate}

\setcounter{enumi}{\value{HW}}

\item $\csc(5.902) \approx -2.688$
\item $\tan(39.672^{\circ}) \approx 0.829$

\setcounter{HW}{\value{enumi}}

\end{enumerate}

\end{multicols}

\begin{multicols}{2}

\begin{enumerate}

\setcounter{enumi}{\value{HW}}

\item $\cot(3^{\circ}) \approx 19.081$
\item $\sec(0.45) \approx 1.111$

\setcounter{HW}{\value{enumi}}

\end{enumerate}

\end{multicols}

\begin{enumerate}

\setcounter{enumi}{\value{HW}}

\item $\tan(\theta) = \sqrt{3}$ when $\theta = \dfrac{\pi}{3} + \pi k$ for any integer $k$
\item $\sec(\theta) = 2$ when $\theta = \dfrac{\pi}{3} + 2\pi k$ or $\theta = \dfrac{5\pi}{3} + 2\pi k$ for any integer $k$
\item $\csc(\theta) = -1$ when $\theta = \dfrac{3\pi}{2} + 2\pi k$ for any integer $k$.
\item $\cot(\theta) = \dfrac{\sqrt{3}}{3}$ when $\theta = \dfrac{\pi}{3} + \pi k$ for any integer $k$
\item $\tan(\theta) = 0$ when $\theta = \pi k$ for any integer $k$
\item $\sec(\theta) = 1$ when $\theta = 2\pi k$ for any integer $k$
\item $\csc(\theta) = 2$ when $\theta = \dfrac{\pi}{6} + 2\pi k$ or $\theta = \dfrac{5\pi}{6} + 2\pi k$ for any integer $k$.
\item $\cot(\theta) = 0$ when $\theta = \dfrac{\pi}{2} + \pi k$ for any integer $k$
\item $\tan(\theta) = -1$ when $\theta = \dfrac{3\pi}{4} + \pi k$ for any integer $k$
\item $\sec(\theta) = 0$ never happens 
\item $\csc(\theta) = -\dfrac{1}{2}$ never happens
\item  $\sec(\theta) = -1$ when $\theta = \pi + 2\pi k = (2k+1)\pi$ for any integer $k$
\item  $\tan(\theta) = -\sqrt{3}$ when $\theta = \dfrac{2\pi}{3} + \pi k$ for any integer $k$
\item  $\csc(\theta) = -2$ when $\theta = \dfrac{7\pi}{6} + 2\pi k$ or $\theta = \dfrac{11\pi}{6} + 2\pi k$ for any integer $k$
\item  $\cot(\theta) = -1$ when $\theta = \dfrac{3\pi}{4} + \pi k$ for any integer $k$

\setcounter{HW}{\value{enumi}}

\end{enumerate}

\begin{enumerate}

\setcounter{enumi}{\value{HW}}

\item $\cot(t) = 1$ when $t = \dfrac{\pi}{4} + \pi k$ for any integer $k$
\item  $\tan(t) = \dfrac{\sqrt{3}}{3}$ when $t = \dfrac{\pi}{6} + \pi k$ for any integer $k$
\item $\sec(t) = -\dfrac{2\sqrt{3}}{3}$ when $t = \dfrac{5\pi}{6} + 2\pi k$ or $t = \dfrac{7\pi}{6} + 2\pi k$ for any integer $k$
\item $\csc(t) = 0$ never happens 
\item $\cot(t) = -\sqrt{3}$ when $t = \dfrac{5\pi}{6} + \pi k$ for any integer $k$
\item $\tan(t) = -\dfrac{\sqrt{3}}{3}$ when $t = \dfrac{5\pi}{6} + \pi k$ for any integer $k$
\item $\sec(t) = \dfrac{2\sqrt{3}}{3}$ when $t = \dfrac{\pi}{6} + 2\pi k$ or $t = \dfrac{11\pi}{6} + 2\pi k$ for any integer $k$
\item $\csc(t) = \dfrac{2\sqrt{3}}{3}$ when $t = \dfrac{\pi}{3} + 2\pi k$ or $t = \dfrac{2\pi}{3} + 2\pi k$ for any integer $k$

\setcounter{HW}{\value{enumi}}
\end{enumerate}

\begin{enumerate}
\setcounter{enumi}{\value{HW}}

\item  One solution is $g(t) = 3t^2$ and $h(t) = 2\tan(3t)$.

\item  One solution is $g(\theta) = \sec(\theta)$ and $h(\theta) = \tan(\theta)$.

\item  One solution is $g(t) = -\csc(t)$ and $h(t) = \cot(t)$.

\item  One solution is $f(t) = \tan(3t)$ and $g(t) = t$.

\item  One solution is $f(\theta) = 4 \theta$ and $g(\theta) = \tan(\theta)$.

\item  Since $\sec^{2}(\theta) = (\sec(\theta))^2$, one solution is $f(\theta) = \sec(\theta)$ and $g(\theta) = \theta^2$.

\item  One solution is $f(x) = \sin(x)$ and $g(x) = \ln(x)$.

\item  One solution is $f(\theta) = \sec(\theta)$, $g(\theta) = \tan(\theta)$, and $h(\theta) = \ln| \theta|$.

\setcounter{HW}{\value{enumi}}
\end{enumerate}

\begin{enumerate}
\setcounter{enumi}{\value{HW}}
\addtocounter{enumi}{1}

\item  As we zoom in towards $0$, the average rate of change of $\tan(k t)$ approaches $k$.  This is the same trend we observed for $\sin(k t)$ in Section \ref{TheCircularFunctionsSineandCosine} number \ref{sinearcexercise}.

\[ \begin{array}{|r||c|c|c|}  \hline

 T(t) &  [-0.1, 0.1] & [-0.01, 0.01] &[-0.001, 0.001] \\ \hline
 \tan(t)     & \approx 1.0033 & \approx 1 & \approx 1 \\  \hline
 \tan(2t)   & \approx 2.0271 & \approx 2.0003 & \approx 2 \\ \hline
 \tan(3t)   & \approx 3.0933 & \approx 3.0009 & \approx 3  \\  \hline
 \tan(4t)   & \approx 4.2279 & \approx 4.0021 & \approx 4  \\  \hline

\end{array} \]


\setcounter{HW}{\value{enumi}}
\end{enumerate}



\end{document}



\smallskip
\closegraphsfile

\end{document}
